
\chapter{Literature Review}
\label{chap:2}

\section{Definition of Co-operative society: }
\begin{itemize}
    \item A Co-operative society is an autonomous association of persons who voluntarily cooperate for their 
    common economic interest.    
    
    \item Cooperatives are made out of cooperative and based on the values of self-help, self-responsibility, 
    democracy and equality, equity and solidarity.
\end{itemize}

\section{Seven Co-operative features:}
\begin{itemize}
    \item \textbf{Voluntary and Open Membership:} The cooperative is open to all individuals who are willing to accept the responsibilities of membership.
    
    \item \textbf{Democratic Member Control:} The members actively participate in making decisions, and the organization is controlled democratically.
    
    \item \textbf{Economic Participation by Members:} Members contribute equitably to, and democratically control, the capital of the cooperative.
    
    \item \textbf{Autonomy and Independence:} The cooperative is an autonomous, self-help organization controlled by its members.
    
    \item \textbf{Education, Training, and Information:} The cooperative provides education and training to its members to help them contribute effectively to the development of the cooperative.
    
    \item \textbf{Cooperation Among Cooperatives:} Cooperatives work together for mutual benefit and the sustainable development of their communities.
    
    \item \textbf{Concern for Community:} Cooperatives strive for the sustainable development of their communities through policies approved by their members.
\end{itemize}

% //////

\section{History of Cooperative Movements}

\begin{itemize}
    \item \textbf{Shore Porters’ Society:} Established in Aberdeen in 1498, the Shore Porters’ Society claims to be one of the earliest cooperative societies globally \cite{WalesCC}.
    
    \item \textbf{Lennoxtown Friendly Victualling Society:} Founded in 1812, the Lennoxtown Friendly Victualling Society is one of the cooperative formations that emerged in the following decades \cite{Lennoxtown}.
    
    \item \textbf{P.C. Plockboy's Proposal (1659):} In 1659, P.C. Plockboy published a pamphlet outlining a scheme for the formation of economic associations \cite{Sampson1906}.
    
    \item \textbf{Frnwick Weaver's Society (1761):} Possibly the first evidential cooperative, Frnwick Weaver's Society, was established on March 14, 1761, in a cottage in Fenwick, East Ayrshire. Local weavers initiated the cooperative by selling discounted oatmeal from John Walker's front room \cite{Rediff}.
    
    \item \textbf{Cooperative Movement in Europe:} The cooperative movement in Europe is believed to have started in the 19th century, primarily in England and France.
    
    \item \textbf{Robert Owen's Contribution:} Robert Owen, a social worker and reformer, is considered a pioneer of the cooperative movement \cite{Infed}.
    
    \item \textbf{Rochdale Principles (1844):} In 1844, the Rochdale Society of Equitable Pioneers established the 'Rochdale Principles,' which became the foundation for their cooperative's operation, development, and the growth of the modern cooperative movement \cite{Carrell2007}.
\end{itemize}


% ////

\section{Cooperative Movement in India}

\begin{itemize}
    \item \textbf{Derrick Nicholson's Initiative (1892):} In 1892, Derrick Nicholson explored ways to establish institutions supporting the agricultural sector. In response to a severe famine in 1899, he proposed the establishment of cooperative societies to aid in agricultural development \cite{PreserveArticles}.
    
    \item \textbf{Co-operative Societies Act (1904):} The Co-operative Societies Act was enacted in 1904, officially recognizing the cooperative movement in India. Subsequently expanded under the 1912 Act, its objective was to assist rural farmers and artisans by providing both short-term and long-term loans \cite{PunjabRevenue}.
    
    \item \textbf{Models for Credit Societies:} Credit societies were organized based on two models: one for rural areas and another for urban areas, each with distinct features. The initial model was the Reinfusion Model, later evolving into the Schulze Delitzsch Bank Model. Despite the growth of cooperative societies in rural areas, they faced challenges and could not operate effectively for various reasons \cite{NCUA}.
    
    \item \textbf{Defects Leading to Ineffectiveness:}
    \begin{itemize}
        \item Lack of provision for establishing Noncredit Cooperative Societies in rural areas.
        \item Absence of a dedicated central agency for financing and supervising the activities of these societies.
        \item The division of Credit Cooperative Societies into rural and urban types posed a barrier, as no specific arrangements could be made due to the overlapping nature of such classification.
    \end{itemize}
\end{itemize}

% Add your bibliography entries here

% ///////////////////////////////////////////////////////////////////////
\section{Cooperatives in Bangladesh}

\begin{itemize}
    \item \textbf{Historical Evolution:} Cooperatives in Bangladesh have traversed a century, initially focusing on agriculture and subsequently expanding into economic spheres. Recognized constitutionally as a vital sector in the post-independent Bangladesh economy, the Cooperative Society has evolved into a significant social institution. Dr. Akther Hamid Khan has played a pivotal role in its establishment (Scribd).
    
    \item \textbf{National Cooperative Day:} Every year on November 6, Bangladesh observes National Cooperative Day throughout the country. During the 39th National Cooperative Day at the Bangabandhu International Conference Centre in Dhaka, Prime Minister Sheikh Hasina (2010) emphasized the urgency of finalizing a national cooperative policy to invigorate cooperative activities. The government is committed to expanding the cooperative movement for socio-economic and cultural emancipation (Sheikh Hasina, 2010).
    
    \item \textbf{Annual Report 2011 Highlights (Scribd):} According to the annual report in 2011, there has been continuous growth in the number of registered primary cooperatives over the last five years. In the fiscal year 2010-11, the number of primary cooperatives increased to 1,75,839 (4.56\% growth), cooperative members reached 89,54,237 (2.23\% growth), share capital rose to T 512.95 crore (9\% growth), and loan disbursement and collection increased to TK 1638.92 crore and TK 1476.98 crore, respectively.
    
    \item \textbf{Revised Rules and Regulations (google.com.bd):}
    \begin{enumerate}
        \item Citation
        \item Application for registration
        \item Annual return
        \item Information in annual return
        \item Other returns
        \item Fee
    \end{enumerate}
    
    \item \textbf{Survey Findings:} A survey of 41 cooperatives revealed:
    \begin{center}
        \begin{tabular}{|c|c|c|}
            \hline
            \textbf{out of 41} & \textbf{Provide information} & \textbf{Reasons} \\
            \hline
            20 & 100\% & We found the responsible person and they were okay with the questions and cooperative. \\
            \hline
            12 & 40\% & We couldn’t find the management/who knows all the information or maybe scared. \\
            \hline
            9 & 10\% & They were actually scared/ignorant/not cooperative. \\
            \hline
        \end{tabular}
        \end{center}

\end{itemize}

% Add your bibliography entries here
\begin{thebibliography}{9}
    \bibitem{WalesCC} Wales Cooperative Centre.
    \bibitem{Lennoxtown} welcometolennoxtown.co.uk.
    \bibitem{Sampson1906} F. A Sampson, 1906.
    \bibitem{Rediff} pages.rediff.com.
    \bibitem{Infed} infed.org.
    \bibitem{Carrell2007} Severin Carrell, 2007.
    \bibitem{PreserveArticles} preservearticles.com.
    \bibitem{PunjabRevenue} punjabrevenue.nic.in.
    \bibitem{NCUA} National Credit Union Administration.
    \bibitem{Scribd} scribd.com.
    \bibitem{SheikhHasina2010} Sheikh Hasina, 2010.
    % Add more references as needed
\end{thebibliography}