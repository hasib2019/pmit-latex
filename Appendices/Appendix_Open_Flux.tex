For the diffusion process, The equation was solved using a two-dimensional implicit Crank-Nicolson scheme, which is unconditionally stable and second-order accurate in both time and space \citep{crank47}. In the conventional notation, the two-dimensional numerical scheme using central differencing can be written for a uniform Cartesian grid as 
\begin{eqnarray}
\nonumber\left(1+2\mu\right)u^{t+1}_{i,j}-\frac{\mu}{2}\left(u^{t+1}_{i+1,j}+u^{t+1}_{i-1,j}+u^{t+1}_{i,j+1}+u^{t+1}_{i,j-1}\right)\\
=\left(1-2\mu\right)u^{t}_{i,j}+\frac{\mu}{2}\left(u^{t}_{i+1,j}+u^{t}_{i-1,j}+u^{t}_{i,j+1}+u^{t}_{i,j-1}\right),
\end{eqnarray}

\noindent where $u^{t}_{i,j}$ is the value of the parameter undergoing the diffusion ($B_{r}$ in this case) at position $(i, j)$ at time \textit{t}. The von Neumann number on a uniform grid is $\mu=\xi{\Delta}t/\left({\Delta}x\right)^{2}$, where the size of the grid square is $\Delta x$ on each side, and the diffusion coefficient $\xi$ describes the speed at which the mathematical diffusion takes place. When deriving the two-dimensional Crank-Nicolson scheme in spherical coordinates, the von Neumann number is written as $\mu=\xi{\Delta}t/\left(r\Delta\theta\right)^{2}$, where $\Delta\theta=\Delta\phi$, and the cosine is replaced by the central difference of the sine to remain consistent with the discrete nature of the other terms. Care must be taken at the poles, where the central differencing is replaced by forward or backward differencing. To keep second-order accuracy with forward or backward differencing, the series must be carried out to higher-order terms in the derivation. The two-dimensional numerical scheme using central differencing can be written for a uniform spherical grid as
\begin{multline}
\left(1+\mu+\frac{\mu}{\sin^{2}\theta_{i,j}}\right)u^{t+1}_{i,j}-\frac{\mu}{2}\left[1+\frac{\left(\sin\theta_{i+1,j}-\sin\theta_{i-1,j}\right)}{4\sin\theta_{i,j}}\right]u^{t+1}_{i+1,j}\\
 -\frac{\mu}{2}\left[1-\frac{\left(\sin\theta_{i+1,j}-\sin\theta_{i-1,j}\right)}{4\sin\theta_{i,j}}\right]u^{t+1}_{i-1,j}-\frac{\mu}{2}\frac{1}{\sin^{2}\theta_{i,j}}u^{t+1}_{i,j+1}-\frac{\mu}{2}\frac{1}{\sin^{2}\theta_{i,j}}u^{t+1}_{i,j-1}\\
 =\left(1-\mu-\frac{\mu}{\sin^{2}\theta_{i,j}}\right)u^{t+1}_{i,j}+\frac{\mu}{2}\left[1+\frac{\left(\sin\theta_{i+1,j}-\sin\theta_{i-1,j}\right)}{4\sin\theta_{i,j}}\right]u^{t+1}_{i+1,j}\\
 +\frac{\mu}{2}\left[1-\frac{\left(\sin\theta_{i+1,j}-\sin\theta_{i-1,j}\right)}{4\sin\theta_{i,j}}\right]u^{t+1}_{i-1,j}+\frac{\mu}{2}\frac{1}{\sin^{2}\theta_{i,j}}u^{t+1}_{i,j+1}+\frac{\mu}{2}\frac{1}{\sin^{2}\theta_{i,j}}u^{t+1}_{i,j-1}.
 \label{CN Spherical}
\end{multline}

Although it is unconditionally stable, a marching scheme such as this will depend on the value of $\mu$ for its accuracy. A lower choice of $\mu$ will lead to a more accurate solution at the expense of computational resources (i.e., a smaller time step ${\Delta}t$), while a higher value of $\mu$ will arrive at a solution more rapidly but with less accuracy (a larger time step). In this model, the value of the coefficient $\xi$ describes the speed of the mathematical relaxation and, since it does not describe a physical process, can be chosen arbitrarily. Thus the only restriction for this scheme will lie in keeping $\mu$ small for accuracy and assigning either $\xi$ or ${\Delta}t$. It can be seen that when $\mu$ is held constant, any choice for either $\xi$ or ${\Delta}t$ will lead to the same solution. A value of $\mu=1/4$ was chosen, with an arbitrary time step of ${\Delta}t=0.1$ s, and studied several different grid resolutions, with a grid size of 2.5$^\circ$ x 2.5$^\circ$ (72 x 144 grid spaces) on a uniform spherical grid used for the comparisons in this paper. The relaxation was allowed to continue on a sphere of $r=R_{\sun}$ until the difference in magnetic field magnitude between any cell and its neighbor was of order $10^{-1}{\mu}T$.